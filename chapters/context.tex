\chapter{Context: Software in Electrical Engineering }

[first software - allow ee projects; written with the languages we had; fortran
Appendix - all the prog. languages; history]

\section{ Languages }

The choice of which programming languages should be taught in computer science and engineering undergraduate programs varies from place to place. According to \cite{mason2018language}, Java and Python are popular choices in the UK and in Australia. Several articles suggest which should be the base language to be adopted considering students with no previous background in programming. \cite{wang2006practical} suggests PHP and Javascript, leaning toward a practical approach by developing web applications. The IEE Spectrum listed their most popular languages in 2015 (\cite{cass20152015}). In many cases, the object oriented programming model \cite{rumbaugh1991object} is the choice for the lectures. The main concepts are translated into classes and associations, with attributes and methods, generally using the imperative style (statements change the program's state). A deep analysis of the object oriented model is not in the scope of this work, but it is important to notice its wide adoption as a first contact with programming in general - and yet, this teaching formula might have complications (see \cite{kolling1999problem}; this article points out some issues with Objected Oriented programming and the imperative style, such as safety - mutable states may be a bad option in some scenarios - , syntax - some languages, such as C++, may have complex syntax for beginners - , etc).


Many courses neglect the existence of other styles and models, indirectly biasing the students to the same solution - object oriented programming and imperative code.

\section{ Alternative styles and languages }

\section{ How to choose a programming language }

[TODO - intro paragraph]

Type systems are tools for reasoning about programs \cite{pierce2002types}. Each programming language has its own type system, and it guarantees that a well-typed program is free of a certain group of misbehaviours. Type systems help with error detection, abstraction, documentation, language safety (protect its own abstractions of unexpected behaviours) and efficiency.


